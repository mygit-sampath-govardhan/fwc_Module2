\iffalse
\documentclass[journal,10pt,twocolumn]{article}
\usepackage{graphicx}
\usepackage[margin=0.5in]{geometry}
\usepackage[cmex10]{amsmath}
\usepackage{amssymb}
\usepackage{array}
\usepackage{booktabs}
\usepackage{mathtools}
\usepackage{dirtree}
\usepackage{xcolor}
\usepackage{float}
\usepackage[justification=centering,font={rm,md,scriptsize}]{caption}
\usepackage{enumitem}
\usepackage{listings}
\usepackage{mathtools}
\usepackage{fancyvrb}
\usepackage{hyperref}

%Add chapter functionality in IEEEtran class
\newcounter{Chapcounter}
\newcommand\showmycounter{\addtocounter{Chapcounter}{1}\themycounter}
\newcommand{\chapter}[1] 
{ {\centering          
  \addtocounter{Chapcounter}{1} \large \textbf{Chapter \theChapcounter ~#1}}  
  \addcontentsline{toc}{section}{Chapter ~\theChapcounter~~ #1}    
  \setcounter{section}{0}
}
%%%%

\counterwithin{enumi}{section}
\counterwithin{equation}{enumi}
\counterwithin{figure}{enumi}

\renewcommand\thesection{\theChapcounter.\arabic{section}}
%\renewcommand\thesectiondis{\theChapcounter.\arabic{section}}
\newcommand\figref{Fig.~\ref}

\setenumerate{label=\thesection.\arabic*}

\lstset{
  basicstyle=\ttfamily,
  columns=fullflexible,
  frame=single,
  breaklines=true,
  postbreak=\mbox{\textcolor{red}{$\hookrightarrow$}\space},
}

\providecommand{\mbf}{\mathbf}
\providecommand{\pr}[1]{\ensuremath{\Pr\left(#1\right)}}
\providecommand{\qfunc}[1]{\ensuremath{Q\left(#1\right)}}
\providecommand{\sbrak}[1]{\ensuremath{{}\left[#1\right]}}
\providecommand{\lsbrak}[1]{\ensuremath{{}\left[#1\right.}}
\providecommand{\rsbrak}[1]{\ensuremath{{}\left.#1\right]}}
\providecommand{\brak}[1]{\ensuremath{\left(#1\right)}}
\providecommand{\lbrak}[1]{\ensuremath{\left(#1\right.}}
\providecommand{\rbrak}[1]{\ensuremath{\left.#1\right)}}
\providecommand{\cbrak}[1]{\ensuremath{\left\{#1\right\}}}
\providecommand{\lcbrak}[1]{\ensuremath{\left\{#1\right.}}
\providecommand{\rcbrak}[1]{\ensuremath{\left.#1\right\}}}
\newcommand{\sgn}{\mathop{\mathrm{sgn}}}
\providecommand{\abs}[1]{\left\vert#1\right\vert}
\providecommand{\res}[1]{\Res\displaylimits_{#1}} 
\providecommand{\norm}[1]{\left\lVert#1\right\rVert}
\providecommand{\mtx}[1]{\mathbf{#1}}
\providecommand{\mean}[1]{E\left[ #1 \right]}
\providecommand{\fourier}{\overset{\mathcal{F}}{ \rightleftharpoons}}
\providecommand{\ztrans}{\overset{\mathcal{Z}}{ \rightleftharpoons}}
\providecommand{\system}{\overset{\mathcal{H}}{ \longleftrightarrow}}
\newcommand{\solution}{\noindent \textbf{Solution: }}
\newcommand{\cosec}{\,\text{cosec}\,}
\providecommand{\dec}[2]{\ensuremath{\overset{#1}{\underset{#2}{\gtrless}}}}
\newcommand{\myvec}[1]{\ensuremath{\begin{pmatrix}#1\end{pmatrix}}}
\newcommand{\mydet}[1]{\ensuremath{\begin{vmatrix}#1\end{vmatrix}}}
\providecommand{\gauss}[2]{\mathcal{N}\ensuremath{\left(#1,#2\right)}}
\newcommand*{\permcomb}[4][0mu]{{{}^{#3}\mkern#1#2_{#4}}}
\newcommand*{\perm}[1][-3mu]{\permcomb[#1]{P}}
\newcommand*{\comb}[1][-1mu]{\permcomb[#1]{C}}

\let\vec\mathbf

\def\putbox#1#2#3{\makebox[0in][l]{\makebox[#1][l]{}\raisebox{\baselineskip}[0in][0in]{\raisebox{#2}[0in][0in]{#3}}}}
     \def\rightbox#1{\makebox[0in][r]{#1}}
     \def\centbox#1{\makebox[0in]{#1}}
     \def\topbox#1{\raisebox{-\baselineskip}[0in][0in]{#1}}
     \def\midbox#1{\raisebox{-0.5\baselineskip}[0in][0in]{#1}}

\begin{document}

\title{Maximum Likelihood Detection:BPSK}
\author{Sampath Govardhan}

\maketitle

\tableofcontents

\bigskip


\fi

\section{Maximum Likelihood}



\begin{enumerate}



\item Generate equiprobable $X \in \cbrak{1,-1}$.\\

\item Generate 
\begin{equation}
Y = AX+N,
\end{equation}
		where $A = 5$ dB,  and $N \sim \gauss{0}{1}$.\\

\item Plot $Y$ using a scatter plot.\\

\solution   $X$, $Y$ and the scatter plot of $Y$ ($\figref{fig:bpsk_scatter}$) can be generated using the below code,
\begin{lstlisting}
codes/ch3_scatter.py
\end{lstlisting}
\begin{figure}[h]
\centering
\includegraphics[width=\columnwidth]{./chapters/ch3/figs/ch3_scatter.png}
\caption{Scatter plot of $Y$}
\label{fig:bpsk_scatter}
\end{figure}
\item Guess how to estimate $X$ from $Y$.\\
\solution
\begin{equation}
y \dec{1}{-1} 0
\label{eq:bpsk_decision}
\end{equation}
\item
\label{ml-ch4_sim}
Find 
\begin{equation}
	P_{e|0} = \pr{\hat{X} = -1|X=1}
\end{equation}
and 
\begin{equation}
	P_{e|1} = \pr{\hat{X} = 1|X=-1}
\end{equation}\\
\solution Based on the decision rule in \eqref{eq:bpsk_decision},
\begin{flalign*}
	\pr{\hat{X} = -1|X=1} &= \pr{Y < 0|X=1}&\\
	&= \pr{AX + N < 0|X=1}&\\ 
	&= \pr{A + N < 0}&\\
	&= \pr{N < -A}
\end{flalign*}
Similarly,
\begin{flalign*}
	\pr{\hat{X} = 1|X=-1} &= \pr{Y > 0|X=-1}&\\
	&= \pr{N > A}
\end{flalign*}
Since $N \sim \gauss{0}{1}$,
\begin{flalign}
	\label{eq:std_norm_symmetric}
	\pr{N < -A} &= \pr{N > A}&\\
	\label{eq:bpks_prob_err_cond}
	\implies P_{e|0} &= P_{e|1} = \pr{N > A}
\end{flalign}
%
\item Find $P_e$ assuming that $X$ has equiprobable symbols.\\
\solution
\begin{flalign}
	P_e &= \pr{X=1}P_{e|1} + \pr{X=-1}P_{e|0}&\\
	\intertext{Since $X$ is equiprobable}\\
	\label{eq:bpsk_prob_error_equi}
	P_e &= \frac{1}{2}P_{e|1} + \frac{1}{2}P_{e|0}
\end{flalign}
Substituting from \eqref{eq:bpks_prob_err_cond}
\begin{equation}
	P_e = \pr{N > A}
\end{equation}
Given a random varible $X \sim \gauss{0}{1}$ the Q-function is defined as
\begin{align}
	Q(x) &= \pr{X > x}\\
	\label{eq:q_func_integral}
	Q(x) &= \frac{1}{\sqrt{2\pi}} \int_x^\infty \exp\left(-\frac{u^2}{2}\right) \, du.\\
\end{align}
Using the Q-function, $P_e$ is rewritten as
\begin{equation}
	P_e = Q(A)
\end{equation} 
%
\item
Verify by plotting  the theoretical $P_e$ with respect to $A$ from 0 to 10 dB.\\
\solution The theoretical $P_e$ is plotted in $\figref{fig:bpsk_pe_snr}$, along with numerical estimations from generated samples of $Y$. The below code is used for the plot, 
\begin{lstlisting}
codes/chapter3/ch3_varyA.py
\end{lstlisting}
\begin{figure}[h]
\centering
\includegraphics[width=\columnwidth]{./chapters/ch3/figs/ch3_varyA.png}
\caption{$P_e$ versus $A$ plot}
\label{fig:bpsk_pe_snr}
\end{figure}
%
\item Now, consider a threshold $\delta$  while estimating $X$ from $Y$. Find the value of $\delta$ that maximizes the theoretical $P_e$.\\
\label{prob:bpsk_delta_equi}
\solution Given the decision rule, 
\begin{equation}
y \dec{1}{-1} \delta
\label{eq:bpsk_decision_delta}
\end{equation}
\begin{flalign*}
	P_{e|0} &= \pr{\hat{X} = -1|X=1}&\\
	&= \pr{Y < \delta|X=1}&\\
	&= \pr{AX + N < \delta|X=1}&\\ 
	&= \pr{A + N < \delta}&\\
	&= \pr{N < -A + \delta}&\\
	&= \pr{N > A - \delta}&\\
	&= Q(A-\delta)
\end{flalign*}
\begin{flalign*}
	P_{e|1} &= \pr{\hat{X} = 1|X=-1}&\\
	&= \pr{Y > \delta|X=-1}&\\
	&= \pr{N > A + \delta}&\\
	&= Q(A+\delta)
 \end{flalign*}
Using \eqref{eq:bpsk_prob_error_equi}, $P_e$ is given by
\begin{flalign}
	P_e &= \frac{1}{2}Q(A+\delta) + \frac{1}{2}Q(A-\delta)
\end{flalign}
Using the integral for Q-function from \eqref{eq:q_func_integral},
\begin{align}
	\label{eq:prob_error_delta_equi}
	P_e &= k(\int_{A+\delta}^\infty \exp\left(-\frac{u^2}{2}\right) \, du + \int_{A-\delta}^\infty \exp\left(-\frac{u^2}{2}\right) \, du)\\
	\intertext{where k is a constant}	\nonumber
\end{align}
Differentiating \eqref{eq:prob_error_delta_equi} wrt $\delta$ (using Leibniz's rule) and equating to $0$, we get
\begin{flalign*}
	\exp\left(-\frac{(A+\delta)^2}{2}\right)-\exp\left(-\frac{(A-\delta)^2}{2}\right) &= 0&\\
	\frac{\exp\left(-\frac{(A+\delta)^2}{2}\right)}{\exp\left(-\frac{(A-\delta)^2}{2}\right)} &= 1&\\
	\exp\left(-\frac{(A+\delta)^2-(A-\delta)^2}{2}\right) &= 1&\\
	\exp\left(-2A\delta\right) &= 1&\\
	\intertext{Taking $\ln$ on both sides}\\
	-2A\delta &= 0&\\
	\implies \delta &= 0
\end{flalign*}
$P_e$ is maximum for $\delta = 0$
\item Repeat the above exercise when 
\label{prob:bpsk_decision_uneqi}
	\begin{align}
		p_{X}(0) = p
	\end{align}\\
\solution Since $X$ is not equiprobable, $P_e$ is given by,
\begin{flalign*}
	P_e &= (1-p)P_{e|1} + pP_{e|0}&\\
	&= (1-p)Q(A+\delta) + pQ(A-\delta)
\end{flalign*}
Using the integral for Q-function from \eqref{eq:q_func_integral},
\begin{multline}
	\label{eq:prob_error_delta_nonequi}
	P_e = k((1-p)\int_{A+\delta}^\infty \exp\left(-\frac{u^2}{2}\right) \, du + \\
	p\int_{A-\delta}^\infty \exp\left(-\frac{u^2}{2}\right) \, du)
\end{multline}
where $k$ is a constant.\\
Following the same steps as in problem \ref{prob:bpsk_delta_equi}, $\delta$ for maximum $P_e$ evaluates to,
\begin{equation}
	\delta = \frac{1}{2A}\ln\left(\frac{1}{p}-1\right)
\end{equation}
\item Repeat the above exercise using the MAP criterion.\\
\solution 
The MAP rule can be stated as\\
\begin{flalign}
\label{eq:map_rule}
\text{Set } \hat{x} &= x_i \text{ if}&\\ \nonumber
p_X(x_k)p_Y(y|x_k) &\text{ is maximum for } k = i
\end{flalign}
For the case of BPSK, the point of equality between $p_X(x=1)p_Y(y|x=1)$ and $p_X(x=-1)p_Y(y|x=-1)$ is the optimum threshold. If this threshold is $\delta$, then
\begin{flalign*}
	pp_Y(y|x=1) > (1-p)p_Y(y|x=-1) &\text{ when } y > \delta&\\
	pp_Y(y|x=1) < (1-p)p_Y(y|x=-1) &\text{ when } y < \delta 	
\end{flalign*}
The above inequalities can be visualized in below figure for p = 0.3 and A = 3.
\begin{figure}[h]
\centering
\includegraphics[width=\columnwidth]{./chapters/ch3/figs/ch3_map.png}
\caption{$p_X(X=x_i)p_Y(y|x=x_i)$ versus $y$ plot for $X \in \{-1,1\}$}
\label{fig:bpsk_map_density}
\end{figure}
Given $Y=AX+N$ where $N$ $\sim$ $\gauss{0}{1}$, the optimum threshold is found as solution to the below equation
\begin{equation}
	p\exp\left(-\frac{(y_{eq}-A)^2}{2}\right) = (1-p)\exp\left(-\frac{(y_{eq}+A)^2}{2}\right)
\end{equation}
Solving for $y_{eq}$, we get
\begin{equation}
	y_{eq} = \delta = \frac{1}{2A}\ln\left(\frac{1}{p}-1\right)
\end{equation}
which is same as $\delta$ obtained in problem \ref{prob:bpsk_decision_uneqi}

\end{enumerate}
%\end{document}
